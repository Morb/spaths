Questo programma genera grafi casuali orientati debolmente connessi, li disegna tramite interfaccia grafica e visualizza il percorso più breve tra 2 nodi di esso, se esiste.~\newline
 Si invoca da linea di comando senza alcun parametro addizionale\+:~\newline

\begin{DoxyItemize}
\item {\ttfamily }./\+S\+Paths

Il grafo orientato è implementato mediante liste di adiacenza.~\newline
 La ricerca del percorso più breve è effettuata con l'algoritmo di Bellman-\/\+Ford (vedi \hyperlink{operazioni__grafo_8h_a80eb34ff061292b73546e57d066265b6}{bellman\+\_\+ford}). \begin{DoxyParagraph}{Moduli}

\begin{DoxyItemize}
\item {\bfseries \hyperlink{SPaths_8cc}{S\+Paths.\+cc}\+:} ~\newline
 contiene la funzione main, necessaria per l'avvio del programma e gli handler per l'interfaccia grafica. ~\newline
 Utilizza {\ttfamily \hyperlink{strutt__dati_8h}{strutt\+\_\+dati.\+h}}, {\ttfamily \hyperlink{operazioni__grafo_8h}{operazioni\+\_\+grafo.\+h}}, {\ttfamily \hyperlink{disegna__grafo_8h}{disegna\+\_\+grafo.\+h}}, {\ttfamily \hyperlink{operazioni__file_8h}{operazioni\+\_\+file.\+h}} 
\item {\bfseries \hyperlink{strutt__dati_8h}{strutt\+\_\+dati.\+h}\+:} ~\newline
 contiene le strutture dati utilizzate da tutti i moduli.
\item {\bfseries \hyperlink{operazioni__grafo_8cc}{operazioni\+\_\+grafo.\+cc}\+:} ~\newline
 contiene le funzioni utilizzate dal programma per lavorare su tipi di dato grafo. ~\newline
 Utilizza {\ttfamily \hyperlink{strutt__dati_8h}{strutt\+\_\+dati.\+h}} e {\ttfamily \hyperlink{operazioni__grafo_8h}{operazioni\+\_\+grafo.\+h}} 
\item {\bfseries \hyperlink{disegna__grafo_8cc}{disegna\+\_\+grafo.\+cc}\+:} ~\newline
 contiene le funzioni utilizzate dall'interfaccia grafica per disegnare il grafo ed i percorsi minimi. ~\newline
 Utilizza {\ttfamily \hyperlink{strutt__dati_8h}{strutt\+\_\+dati.\+h}}, {\ttfamily \hyperlink{disegna__grafo_8h}{disegna\+\_\+grafo.\+h}} e {\ttfamily \hyperlink{operazioni__grafo_8h}{operazioni\+\_\+grafo.\+h}} 
\item {\bfseries \hyperlink{operazioni__file_8cc}{operazioni\+\_\+file.\+cc}\+:} ~\newline
 contiene le funzioni utilizzate per salvare il grafo corrente su file e per caricarlo successivamente. ~\newline
 Utilizza {\ttfamily \hyperlink{strutt__dati_8h}{strutt\+\_\+dati.\+h}} 
\end{DoxyItemize}
\end{DoxyParagraph}
\begin{DoxySeeAlso}{Si veda anche}
\hyperlink{strutt__dati_8h}{strutt\+\_\+dati.\+h} Per vedere come è stato implementato il \hyperlink{structgrafo}{grafo} 
\end{DoxySeeAlso}
\begin{DoxyAuthor}{Autore}
Federico Terzi 
\end{DoxyAuthor}
\begin{DoxyVersion}{Versione}
1.\+0 
\end{DoxyVersion}

\end{DoxyItemize}